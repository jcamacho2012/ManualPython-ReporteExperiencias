% Documento de Observaciones, Conclusiones y Experiencias del desarrollo
%Proyecto Cinema
% 24/11/2012
%     

\documentclass[letterpaper,11pt]{article}
\usepackage[spanish]{babel}
\usepackage[dvips]{epsfig}  
\usepackage[T1]{fontenc}
\usepackage[latin1]{inputenc}
\usepackage{tabularx}
\usepackage{url}
\usepackage{pslatex}
\usepackage{float}

     \topmargin -1 cm
     \textheight 23 cm
     \hoffset -1.5 cm
     
\pagestyle{empty}
\newcommand{\PreserveBackslash}[1]{\let\temp=\\#1\let\\=\temp}  
\let\PBS=\PreserveBackslash
%\renewcommand{\familydefault}{ccr}
 

\usepackage[dvips]{graphicx} % LaTeX
 \usepackage{graphicx}
\begin{document}

%%%%%%% Titulo
\begin{tabular}{c >{\PBS\centering\vspace{-2,7cm}}p{11,5cm} c}
  & \rule{11,5cm}{0.7pt} \newline \newline
 \textbf{\textsc{{\Large Reporte de Experiencias de Proyecto de Lenguajes de Programaci\'on \newline\newline Jos\'e Camacho Mart\'inez} }}\newline
 \rule{11,5cm}{0.7pt} & \newline 
 %\epsfig{file=mave.eps,width=70pt}
\end{tabular} 



%%%%% ESTUDIOS
\section{Proyecto ANDROID - Cinema}

\begin{tabular}{lll}

La experiencia final del proyecto fue muy gratificante ya que nos permiti\'o conocer una nueva \\
plataforma de desarrollo que esta en pleno crecimiento en el mercado mundial en cuesti\'on de \\
desarrollo de aplicaciones.\\\\
La experiencia en ciertas partes del c\'odigo fue algo preocupante, debido a que como principiantes \\
en el desarrollo desconoc\'iamos como esta estructurado un proyecto ANDROID.\\\\
Adem\'as tuvimos problemas en hacer la conexi\'on a una base de datos externa e interna, las que\\
finalmente se pudo hacerlas.\\ \\
Lo bueno de programar en ambiente ANDROID es que es en lenguaje java, que es ya conocido\\
por nosotros, por haber tomado la materia PROGRAMACION ORIENTADA A OBJETOS.\\\\


\end{tabular}

%%%%% CURSOS Y SEMINARIOS
\section{Proyecto PYTHON - Cuentos de Navidad}

\begin{tabular}{lll}

La experiencia final de proyecto fue muy buena en la parte de la librer\'ia que se uso y del\\
ambiente de programaci\'on que ofrece PYTHON.\\\\
Para el proyecto se utiliz\'o la librer\'ia PYGAME, que es una librer\'ia de PYTHON que permite la\\
creaci\'on de videojuegos o aplicaciones gr\'aficas en dos dimensiones. Esta librer\'ia fue de gran\\
ayuda ya que me permiti\'o manejar la interfaz y los sonidos de manera sencilla.\\\\
En lo del ambiente de programaci\'on no fue gran problema el c\'odigo, la l\'ogica es casi la misma\\
que los demas lenguajes de programaci\'on como JAVA, C++, etc. Lo \'unico nuevo era saber que\\
en PYTHON es bien estricto en la estructura de las funciones, espaciado de las l\'ineas de c\'odigo y\\
no hay tipos de datos como entero, string, etc. Todos son objetos.\\\\
Lo \'unico dif\'icil fue como PYTHON maneja los eventos del teclado, pero se encontr\'o\\
r\'apidamente la soluci\'on.\\\\

\end{tabular}

%%%%% ENLACES DE AYUDA
\section{Proyecto HASKELL - MasterMind}

\begin{tabular}{lll}

La experiencia final de proyecto fue buena, algo complicada al principio por el hecho de\\
implementar soluciones que fueron propuestas por personas que investigaron este juego.\\\\
El trabajar con programas de tipo funcional fue complicado ya que era un tipo de\\
programaci\'on nuevo para mi y adem\'as la forma de como definir la sintaxis era diferente\\
de los otros lenguajes de programaci\'on visto en las materias dictadas en FIEC para la carrera\\
de computaci\'on.\\\\
Adem\'as tuve varios problemas buscando ejemplos de c\'odigo, porque en la web hay\\
ejemplos pero muy puntuales, es decir destinados para principiantes (como el t\'ipico "Hola\\
mundo). Por esto pienso que este proyecto fue el m\'as complicado que los anteriores.\\\\

\end{tabular}


\end{document}
